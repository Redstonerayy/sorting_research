\documentclass[25pt, a0paper, portrait]{tikzposter}

% package
\usepackage{graphicx}
\usepackage{fancyhdr}
\usepackage{xcolor}
\usepackage{bchart}
\usepackage{tikz}
\usepackage{multicol}

% language
\usepackage[german]{babel}
\usetikzlibrary{babel}

% font to helvec
% \usepackage{unicode-math}
\usepackage{fontspec}
\setmainfont{Arial}

% custom theming
% https://github.com/debjyoti385/tikzposter

% define colorstyle
\definecolorstyle{Rodenland}{
    % Define default colors
    % PurpleGrayBlue
    \definecolor{colorOne}{HTML}{AE0D45}
    \definecolor{colorTwo}{HTML}{7F8897}
    \definecolor{colorThree}{HTML}{C8512D}
}{
     % Background Colors
    \colorlet{backgroundcolor}{white}
    \colorlet{framecolor}{white}
    % Title Colors
    \colorlet{titlebgcolor}{colorOne}
    \colorlet{titlefgcolor}{white}
    % Block Colors
    \colorlet{blocktitlebgcolor}{colorTwo}
    \colorlet{blocktitlefgcolor}{colorOne}
    \colorlet{blockbodybgcolor}{white}
    \colorlet{blockbodyfgcolor}{black}
    % Innerblock Colors
    \colorlet{innerblocktitlebgcolor}{colorThree}
    \colorlet{innerblocktitlefgcolor}{white}
    \colorlet{innerblockbodybgcolor}{white}
    \colorlet{innerblockbodyfgcolor}{black}
    % Note colors
    \colorlet{notefgcolor}{black}
    \colorlet{notebgcolor}{colorOne!20!white}
    \colorlet{notefrcolor}{colorOne!00!white}
 }

 % define block stzle
 \defineblockstyle{Minimal}{
    titlewidthscale=1, bodywidthscale=1, titleleft,
    titleoffsetx=0pt, titleoffsety=0pt, bodyoffsetx=0pt, bodyoffsety=0pt,
    bodyverticalshift=0pt, roundedcorners=0, linewidth=0.2cm,
    titleinnersep=1cm, bodyinnersep=1cm
}{
    \begin{scope}[line width=\blocklinewidth, rounded corners=\blockroundedcorners]
       \ifBlockHasTitle %
           \draw[draw=none]%, fill=blockbodybgcolor]
               (blockbody.south west) rectangle (blocktitle.north east);
        %    \draw[color=blocktitlebgcolor, loosely dashed]
        %        (blocktitle.south west) -- (blocktitle.south east);%
       \else
             \draw[draw=none]%, fill=blockbodybgcolor]
                 (blockbody.south west) rectangle (blockbody.north east);
        \fi
    \end{scope}
}

% inner block style
\defineinnerblockstyle{Default}{
    titlewidthscale=1, bodywidthscale=1, titlecenter,
    titleoffsetx=0pt, titleoffsety=0pt, bodyoffsetx=0pt, bodyoffsety=0pt,
    bodyverticalshift=0pt, roundedcorners=20, linewidth=4pt,
    titleinnersep=12pt, bodyinnersep=12pt
}{
    \begin{scope}[line width=\innerblocklinewidth, rounded
      corners=\innerblockroundedcorners, solid]
        \ifInnerblockHasTitle %
           \draw[color=innerblocktitlebgcolor, fill=innerblocktitlebgcolor]
           (innerblockbody.south west) rectangle (innerblocktitle.north east);
           \draw[color=innerblocktitlebgcolor, fill=innerblockbodybgcolor]
           (innerblockbody.south west) rectangle (innerblockbody.north east);
        \else
           \draw[color=innerblocktitlebgcolor, fill=innerblockbodybgcolor]
           (innerblockbody.south west) rectangle (innerblockbody.north east);
        \fi
    \end{scope}
}

 % define layout theme
\definelayouttheme{PosterAnton}{
    \usecolorstyle{Rodenland}
    \usebackgroundstyle{Default}
    \usetitlestyle{Default}
    \useblockstyle{Minimal}
    \useinnerblockstyle{Default}
    \usenotestyle{Default}
}

% Theme Simple
\usetheme{PosterAnton}

% change background color of poster
\colorlet{backgroundcolor}{white}


% main document
% \subsection{Quicksort Implementationen in Python mit verschiedenen Optimierungen und Vergleich mit C++}
\begin{document}

% title
\title{Supersonic Algorithms}
\author{Anton Rodenwald (18), Schillerschule Hannover}

\maketitle

% content
% \block{Quicksort Implementationen in Python mit verschiedenen Optimierungen und Vergleich mit C++}{}

\block{Quicksort Implementationen in Python mit verschiedenen Optimierungen und Vergleich mit C++}{
    \begin{multicols}{2}
        \Large asdfasdf asdf as fasdf asdf asdf asdf asd
        % \hspace{20cm}
        \columnbreak
        \begin{tikzfigure}[]
            \hspace{-5cm}\begin{bchart}[min=0, max=80, scale=5]
                \bcbar[label=1, text=Quicksort mit NumPy Array]{72.231}
                \smallskip
                \bcbar[label=2, text=Quicksort mit Cython kompiliert]{52.808}
                \smallskip
                \bcbar[label=3, text=Quicksort mit Python Liste]{40.493}
                \smallskip
                \bcbar[label=4, text=\hspace{12cm}Quicksort mit Cython (alle Optimierungen)]{14.717}
                \smallskip
                \bcbar[label=5, text=\hspace{12cm}Quicksort mit Cython (statische Typisierung)]{14.403}
                \smallskip
                \bcbar[label=7, text=\hspace{12cm}Quicksort mit Numba und numba.typed list]{9.744}
                \smallskip
                \bcbar[label=7, text=\hspace{7cm}list.sort() mit Python Liste]{5.314}
                \smallskip
                \bcbar[label=8, text=\hspace{7cm}Quicksort mit Numba und NumPy Array]{1.805}
                \smallskip
                \bcbar[label=9, text=\hspace{7cm}C++ Quicksort]{1.051}
                \smallskip
                \bcbar[label=12, text=\hspace{7cm}C Quicksort aufgerufen mit CTypes]{1.035}
                \smallskip
                \bcbar[label=11, text=\hspace{7cm}C++ Standard Quicksort (std::sort)]{0.909}
                \smallskip
                \bcbar[label=12, text=\hspace{7cm}list.sort() mit NumPy Array]{0.894}
                \smallskip
                \bcxlabel{\bf{Ausführungszeit in Sekunden}}
            \end{bchart}
        \end{tikzfigure}
    \end{multicols}
}

\end{document}
