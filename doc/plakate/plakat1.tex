\documentclass[25pt, a0paper, portrait]{tikzposter}

% package
\usepackage{graphicx}
\usepackage{fancyhdr}
\usepackage{xcolor}
\usepackage{bchart}
\usepackage{tikz}
\usepackage{multicol}
\usepackage{enumitem}
\usepackage{setspace}
\usepackage{dirtytalk}

% language
\usepackage[german]{babel}
\usetikzlibrary{babel}

% font to helvec
\usepackage{unicode-math}
\usepackage{fontspec}
\setmainfont{Arial}

% custom theming
% https://github.com/debjyoti385/tikzposter

% define colorstyle
\definecolorstyle{Rodenland}{
    % Define default colors
    % PurpleGrayBlue
    \definecolor{colorOne}{HTML}{AE0D45}
    \definecolor{colorTwo}{HTML}{7F8897}
    \definecolor{colorThree}{HTML}{C8512D}
}{
     % Background Colors
    \colorlet{backgroundcolor}{white}
    \colorlet{framecolor}{white}
    % Title Colors
    \colorlet{titlebgcolor}{colorOne}
    \colorlet{titlefgcolor}{white}
    % Block Colors
    \colorlet{blocktitlebgcolor}{colorTwo}
    \colorlet{blocktitlefgcolor}{colorOne}
    \colorlet{blockbodybgcolor}{white}
    \colorlet{blockbodyfgcolor}{black}
    % Innerblock Colors
    \colorlet{innerblocktitlebgcolor}{colorThree}
    \colorlet{innerblocktitlefgcolor}{white}
    \colorlet{innerblockbodybgcolor}{white}
    \colorlet{innerblockbodyfgcolor}{black}
    % Note colors
    \colorlet{notefgcolor}{black}
    \colorlet{notebgcolor}{colorOne!20!white}
    \colorlet{notefrcolor}{colorOne!00!white}
 }

 \definetitlestyle{Default}{
    width=500mm, roundedcorners=30, linewidth=0.4cm, innersep=1cm,
    titletotopverticalspace=15mm, titletoblockverticalspace=20mm,
    titlegraphictotitledistance=10pt, titletextscale=1
}{
    \begin{scope}[line width=\titlelinewidth, rounded corners=\titleroundedcorners]
        \draw[color=framecolor, fill=titlebgcolor]%
        (\titleposleft,\titleposbottom) rectangle (\titleposright,\titlepostop);
    \end{scope}
}

 % define block stzle
 \defineblockstyle{Minimal}{
    titlewidthscale=1, bodywidthscale=1, titleleft,
    titleoffsetx=0pt, titleoffsety=0pt, bodyoffsetx=0pt, bodyoffsety=0pt,
    bodyverticalshift=0pt, roundedcorners=0, linewidth=0.2cm,
    titleinnersep=1cm, bodyinnersep=1cm
}{
    \begin{scope}[line width=\blocklinewidth, rounded corners=\blockroundedcorners]
       \ifBlockHasTitle %
           \draw[draw=none]%, fill=blockbodybgcolor]
               (blockbody.south west) rectangle (blocktitle.north east);
        %    \draw[color=blocktitlebgcolor, loosely dashed]
        %        (blocktitle.south west) -- (blocktitle.south east);%
       \else
             \draw[draw=none]%, fill=blockbodybgcolor]
                 (blockbody.south west) rectangle (blockbody.north east);
        \fi
    \end{scope}
}

% inner block style
\defineinnerblockstyle{Default}{
    titlewidthscale=1, bodywidthscale=1, titlecenter,
    titleoffsetx=0pt, titleoffsety=0pt, bodyoffsetx=0pt, bodyoffsety=0pt,
    bodyverticalshift=0pt, roundedcorners=20, linewidth=4pt,
    titleinnersep=12pt, bodyinnersep=12pt
}{
    \begin{scope}[line width=\innerblocklinewidth, rounded
      corners=\innerblockroundedcorners, solid]
        \ifInnerblockHasTitle %
           \draw[color=innerblocktitlebgcolor, fill=innerblocktitlebgcolor]
           (innerblockbody.south west) rectangle (innerblocktitle.north east);
           \draw[color=innerblocktitlebgcolor, fill=innerblockbodybgcolor]
           (innerblockbody.south west) rectangle (innerblockbody.north east);
        \else
           \draw[color=innerblocktitlebgcolor, fill=innerblockbodybgcolor]
           (innerblockbody.south west) rectangle (innerblockbody.north east);
        \fi
    \end{scope}
}

 % define layout theme
\definelayouttheme{PosterAnton}{
    \usecolorstyle{Rodenland}
    \usebackgroundstyle{Default}
    \usetitlestyle{Default}
    \useblockstyle{Minimal}
    \useinnerblockstyle{Default}
    \usenotestyle{Default}
}

% Theme Simple
\usetheme{PosterAnton}

% change background color of poster
\colorlet{backgroundcolor}{white}

% remove tikzposter notice at bottom
\tikzposterlatexaffectionproofoff

\settitle{ \centering \vbox{
    \@titlegraphic \\[\TP@titlegraphictotitledistance] \centering
    \color{titlefgcolor} {\bfseries \Huge \sc \textbf{\@title} \par}
    \vspace*{1em}
    {\huge \textbf{\@author} \par} \vspace*{1em} {\LARGE \@institute}
}}

% formatting
\setlength\columnsep{3cm}
\setstretch{1.25}

% redefine title
\makeatletter
\newcommand\insertlogoi[2][]{\def\@insertlogoi{\includegraphics[#1]{#2}}}
\newcommand\insertlogoii[2][]{\def\@insertlogoii{\includegraphics[#1]{#2}}}
\newlength\LogoSep
\setlength\LogoSep{0pt}

\insertlogoi[width=5cm]{example-image-a}
\insertlogoii[width=5cm]{example-image-b}

\renewcommand\maketitle[1][]{  % #1 keys
    \normalsize
    \setkeys{title}{#1}
    % Title dummy to get title height
    \node[transparent,inner sep=\TP@titleinnersep, line width=\TP@titlelinewidth, anchor=north, minimum width=\TP@visibletextwidth-2\TP@titleinnersep]
        (TP@title) at ($(0, 0.5\textheight-\TP@titletotopverticalspace)$) {\parbox{\TP@titlewidth-2\TP@titleinnersep}{\TP@maketitle}};
    \draw let \p1 = ($(TP@title.north)-(TP@title.south)$) in node {
        \setlength{\TP@titleheight}{\y1}
        \setlength{\titleheight}{\y1}
        \global\TP@titleheight=\TP@titleheight
        \global\titleheight=\titleheight
    };

    % Compute title position
    \setlength{\titleposleft}{-0.5\titlewidth}
    \setlength{\titleposright}{\titleposleft+\titlewidth}
    \setlength{\titlepostop}{0.5\textheight-\TP@titletotopverticalspace}
    \setlength{\titleposbottom}{\titlepostop-\titleheight}

    % Title style (background)
    \TP@titlestyle

    % Title node
    \node[inner sep=\TP@titleinnersep, line width=\TP@titlelinewidth, anchor=north, minimum width=\TP@visibletextwidth-2\TP@titleinnersep]
        at (0,0.5\textheight-\TP@titletotopverticalspace)
        (title)
        {\parbox{\TP@titlewidth-2\TP@titleinnersep}{\TP@maketitle}};

    % \node[inner sep=0pt,anchor=west] 
    %   at ([xshift=-\LogoSep]title.west)
    %   {\@insertlogoi};

    \node[inner sep=0pt,anchor=east, right=-8cm] 
      at ([xshift=\LogoSep]title.east)
      {\@insertlogoii};

    % Settings for blocks
    \normalsize
    \setlength{\TP@blocktop}{\titleposbottom-\TP@titletoblockverticalspace}
}
\makeatother

% main document
\begin{document}

% title with logo
\insertlogoii[width=5cm]{apple.jpg}

\title{Supersonic Algorithms}
\author{Anton Rodenwald (18), Schillerschule Hannover}

\maketitle

\block[bodyoffsety=4cm]{Quicksort Implementationen in Python mit verschiedenen Optimierungen und Vergleich mit C++}{}
\begin{columns}
    \column{0.5}
    \block{}{
        \begin{tikzfigure}[]
            \Large
            \setlength\multicolsep{0cm}
            \begin{itemize}
                \item Ausführungszeit meiner verschiedenen Implementationen des
                      Sortieralgorithmus Quicksort in Python und C++
                \item Programme sortierten 10 Millionen vorher generierte, zufällige, unsortierte Ganzzahlen
                      in auf- bzw. absteigende Reihenfolge
                \item Erste Version aus dem Informatikunterricht (3) ca. 41 Sekunden
                \item Optimierungsversuche überraschendeweise teils langsamer (1, 2)
                \item Optimierungen mit Cython (4, 5) bringen deutlichen Performance Boost
                \item Versionen mit Numba sogar nochmals schneller (6, 8)
                \item Eine Version (8) sogar schneller als list.sort(), der standardmäßigen Python Sortierfunktion
                \item Versionen mit C++, CTypes und NumPy Arrays nochmals eine Stufe darüber (9 - 12)
                \item Version mit CTypes benötigt allerdings noch zusätzlich Konvertierungszeit von ca. 2 Sekunden
                \item Python ähnlich schnell wie C++, was ich so nicht erwartet hatte
            \end{itemize}
        \end{tikzfigure}
    }

    \column{0.5}
    \block{}{
        \begin{tikzfigure}[]
            \hspace{-5cm}\begin{bchart}[min=0, max=80, scale=3.7]
                \bcbar[label=1, text=Quicksort mit NumPy Array]{72.231}
                \smallskip
                \bcbar[label=2, text=Quicksort mit Cython kompiliert]{52.808}
                \smallskip
                \bcbar[label=3, text=Quicksort mit Python Liste]{40.493}
                \smallskip
                \bcbar[label=4, text=\hspace{12cm}Quicksort mit Cython (alle Optimierungen)]{14.717}
                \smallskip
                \bcbar[label=5, text=\hspace{12cm}Quicksort mit Cython (statische Typisierung)]{14.403}
                \smallskip
                \bcbar[label=6, text=\hspace{12cm}Quicksort mit Numba und numba.typed list]{9.744}
                \smallskip
                \bcbar[label=7, text=\hspace{7cm}list.sort() mit Python Liste]{5.314}
                \smallskip
                \bcbar[label=8, text=\hspace{7cm}Quicksort mit Numba und NumPy Array]{1.805}
                \smallskip
                \bcbar[label=9, text=\hspace{7cm}C++ Quicksort]{1.051}
                \smallskip
                \bcbar[label=10, text=\hspace{7cm}C Quicksort aufgerufen mit CTypes]{1.035}
                \smallskip
                \bcbar[label=11, text=\hspace{7cm}C++ Standard Quicksort (std::sort)]{0.909}
                \smallskip
                \bcbar[label=12, text=\hspace{7cm}list.sort() mit NumPy Array]{0.894}
                \smallskip
                \bcxlabel{\bf{Ausführungszeit in Sekunden}}
            \end{bchart}
        \end{tikzfigure}
    }
\end{columns}


\block[bodyoffsety=4cm]{Quicksort Implementationen in weiteren Sprachen im Vergleich}{}
\begin{columns}
    \column{0.5}
    \block{}{
        \Large
        \begin{tikzfigure}[]
            \begin{bchart}[min=0, max=15, scale=3.5]
                \bcbar[label=1, text=Quicksort Lua]{11.528}
                \smallskip
                \bcbar[label=2, text=Lua table.sort (Standardsortierung)]{11.035}
                \smallskip
                \bcbar[label=3, text=\hspace{14cm}Java Collection.sort (Standardsortierung)]{5.016}
                \smallskip
                \bcbar[label=4, text=\hspace{14cm}Quicksort Javascript]{3.167}
                \smallskip
                \bcbar[label=5, text=\hspace{14cm}Quicksort Java]{2.828}
                \smallskip
                \bcbar[label=6, text=\hspace{7cm}Quicksort Julia und Julia sort (Standardsortierung)]{1.374}
                \smallskip
                \bcbar[label=7, text=\hspace{7cm}Javascript array.sort (Standardsortierung)]{1.159}
                \smallskip
                \bcbar[label=8, text=\hspace{7cm}Quicksort Go (7 Millionen Zahlen)]{0.919}
                \smallskip
                \bcxlabel{\bf{Ausführungszeit in Sekunden}}
            \end{bchart}
        \end{tikzfigure}
    }

    \column{0.5}
    \block{}{
        \Large
        \begin{itemize}
            \item Auch Implementierte ich Quicksort in anderen Programmiersprachen,
                  um deren Geschwindigkeit zu erforschen
            \item Nicht besonders optimiert, da ich kaum Erfahrung in diesen Sprachen habe
            \item Lua relativ langsam (1, 2)
            \item Bereitgestellte Sortierfunktion von Java (3) komischerweise langsamer als
                  eigene Quicksort Implementation in Java (5)
            \item Javascript Quicksort im Mittelfeld (4)
            \item Standardsortierung und eigene Implementation in Julia ähnlich schnell (6)
            \item Javascript array.sort am schnellsten (7)
            \item Go Quicksort Implementation zwar über Javascript, sortierte aber nur 7 Millionen
                  Zahlen aufgrund von Implementationsproblemen
        \end{itemize}
    }

\end{columns}


\block[bodyoffsety=4cm]{C++ Radixsort Implementationen}{}
\begin{columns}
    \column{0.5}
    \block{}{
        \Large
        \begin{itemize}
            \item Versuch der möglichst schnellen Sortierung mit einem anderem Algorithmus (Radixsort)
            \item Nicht direkt mit der Quicksort vergleichbar, da anderer Algorithmus
            \item Erste Version noch relativ langsam (1), da ich Implementationsfehler machte
            \item Verbesserte Implementationen (2, 3) deutlich schneller
            \item AVX2 (\say{Advanced Vector Instructions} , spezielle Befehle für die CPU) als Optimierungsmöglichkeit genutzt
        \end{itemize}
    }

    \column{0.5}
    \block{}{
        \begin{tikzfigure}[]
            \begin{bchart}[min=0, max=2.5, scale=3.7]
                \bcbar[label=1, text=Radixsort mit Basis 10 (Countingsort)]{2.269}
                \smallskip
                \bcbar[label=2, text=\hspace{7cm}Radixsort mit Basis 256 (Bytesort)]{0.244}
                \smallskip
                \bcbar[label=3, text=\hspace{7cm}Radixsort mit Basis 256 (Bytesort) und AVX2]{0.240}
                \smallskip
                \bcxlabel{\bf{Ausführungszeit in Sekunden}}
            \end{bchart}
        \end{tikzfigure}
    }

\end{columns}


\end{document}
